\documentclass{article}

% Language setting
% Replace `english' with e.g. `spanish' to change the document language
\usepackage[english]{babel}

% Set page size and margins
% Replace `letterpaper' with `a4paper' for UK/EU standard size
\usepackage[letterpaper,top=2cm,bottom=2cm,left=3cm,right=3cm,marginparwidth=1.75cm]{geometry}

% Useful packages
\usepackage{amsmath}
\usepackage{graphicx}
\usepackage[colorlinks=true, allcolors=blue]{hyperref}

\usepackage{titling}
\newcommand{\subtitle}[1]{%
  \posttitle{%
    \par\end{center}
    \begin{center}\large#1\end{center}
    \vskip0.5em}%
}

\title{Comparison of solution search algorithms, for an optimal fitness plan for a pet (the ``lazy animal'' problem)}
\subtitle{Introduction to A.I. (CSCI-4511W) - Final Project}
\author{Ryan Jackson \\ jack1805@umn.edu \\ University of Minnesota}
\date{December 16, 2022}

\begin{document}
\maketitle

\begin{abstract}
    Dan Challou 16 hours ago
If you stayed on track, some parts are reusable (references, discussion of problem , references relationship to problem), and there should be a  portion that is updated and new - discussion on experimental design, results, analysis, summary 
\end{abstract}

\section{Problem Description}

For my CSCI-4511W final project, I will be utilizing several artificial intelligence (A.I.) techniques to address the problem of optimal pet (animal) health, in terms of physical activity. This is a problem because some pets are not very physically active, and both for their well-being, and our (human beings') own reasons, we want them to live as long as reasonably possible. Since there are many factors that go into this, my initial idea was to use a pet cat (agent/actor, real or simulated) and an adversarial A.I. (Mini-Max, etc.) to generate a strategy to keep the cat fit using small rewards for physical activity (a small treat for traveling a number of square feet). In exploring this idea, I realized that it may work, but several other approaches may also work. Therefore my current idea for the project is to run several simulations, using different A.I. techniques, and to compare the results of those simulations (using performance measures, and more abstract qualitative measurements).

I will aim to have the simulations working cross-platform (Mac, Windows, etc.), and the results available as a report write-up, complete with graphs and charts when applicable. I welcome feedback, and hope to expand on this idea in the future.

I will be flexible in my formulation to the problem. For example, instead of a cat, the pet may be a dog or any animal that fits the expected profile. However, if it veers too much, I will have to exclude that pet/agent, and make it clear why it is not viable for the problem formulation. I have compiled several domain data sources to assist with the building and solving the problem, such as the following:
\begin{itemize}
    \item ``Diet and exercise patterns in pet dogs.'' (Article) \cite{slater1995diet}
    \item ``Varram'' - product doing something similar to the idea for this project: https://varram.com \cite{varram}
    \textit{They do not cite any sort of technical reference, and one would have to infer/reverse engineer how this robot is working}
    \item A difference example of pet AI - computer vision to detect animal pain: ``Facial expressions of pain in cats: the development and validation of a Feline Grimace Scale''  \cite{evangelista2019}
\end{itemize}

\section{Addressing the Problem with A.I. Techniques}

The core goal of this project idea is to create an adversarial AI, to generate a strategy, to keep a pet as healthy as possible. This assumes a pet which does not want to get much exercise, and an owner that does want the pet to get an optimal amount of exercise.

Through iterative experimentation, several specific approaches will be tried, and the results will be compared. Below are the current ideas I have for these:
\begin{itemize}
    \item Mini-Max (with and without A-B pruning)
    \item Genetic algorithms (2 forms)
    \item Reinforcement learning
    \item Monte-Carlo Tree Search
    \item Random activity from both agents - serves as a control group
\end{itemize}

\subsection{Software}
  I aim to make good use of the AIMA code we studied in this course, as well as open source software, and custom software I have written to enact the experiments planned for this project. Python will be used as the programming language, and open source libraries will include MushroomRL for reinforcement learning.\cite{mushroom}

\section{Planned Experiments and Observation Approach}

To run the experiments and analyze my results, I plan on using custom Python programs, written to simulate and solve the subject problem, and the results will be analyzed by me, and detailed in an accompanying written report. I realize it is possible to use tools like Jupyter Notebooks to create a document with executable code, but I am not very familiar with this tool, and will try it before committing to it for use with this project.

The overall goal of my choice of problem for the project, is to solve a real-world problem, in a creative way, and explore why or why not such a strategy would work. In the case of "why not" I seek to describe in as much detail and experimental result data, why a specific approach would not work theoretically or in the real world.

\subsection{Performance Measures}
\textit{The `P' in PEAS}

\underline{Performance Measures, for measuring the success of a specific algorithm}
\begin{itemize}
  \item Is the move count of agent 1 greater than 0 and less than the over-fitness parameter? The higher this number in this range, the greater the performance.
  \item What is the move count of agent 2? Presume that a minimal move count is better.
  \item If breaking, or adding, food is an option in the game, assume that a lower count for this action is better.
  \item Assume that agent 1 (cat) can smell the air an unlimited number of times.
\end{itemize}

\underline{Performance Measures, for comparing different algorithms}
\begin{itemize}
  \item Method 1: Value Iteration Algorithm (RL) \cite{mohan2014}
  \textit{Time to converge}
  \item Method 2: Processor utilization
  \item Method 3: Memory requirements
\end{itemize}

Please note: Performance metrics for running simulation(s) should be noted, but not conflated with those that are collected and analyzed for generating the strategy for the adversarial A.I. agent.

\section{Analysis}

From establishing Python code examples, using the AIMA book's code as a base to build upon, I got different results from each of the algorithms, which leads me to believe that I either: didn't structure the problem appropriately or didn't run the experiments (by way of the custom code) enough (iteration number was too low). I initially thought that a "lazy animal" coupled with a human as a "competitor" would make for a great MiniMax problem/game, but going through the aforementioned programming experience, writing out the problem, and talking it over with peers, I have come to the conclusion that this is not the best approach. Though there may be several viable approaches, I believe the best to be reinforcement learning (RL) for the following several reasons:

\begin{itemize}
  \item Benefit 1
  \item Benefit 2
  \item Benefit 3
\end{itemize}

Note that RL can be initialized with "demonstrations" purposeful training examples of a prescribed method on how to perform a task. In this case, a fitness plan solution could be provided by current or former pet owners, as well as crowdsourced.
  
\section{Explanation of References}
As the needs be, I will reference the official textbook(s) for this course, as well as domain information outlined above and on the Web, and official documentation for open source and other programming tools necessary to build the executable portion of the project.

In addition to the preceding types of resources, if possible, I will make use of a pet activity tracker in the real-world, and run tests with my own pet dog - if performed, these types of tests (experiments) would use less computational resources, as simulation would not be necessary. However, it is worth noting that they would/will take longer than simulations to run. The advantage to these types of tests however, is more domain-accurate results (data), which is good for drawing conclusions, and performing future research.

\newpage
\bibliographystyle{plain}
\raggedright
\bibliography{main}

\end{document}
